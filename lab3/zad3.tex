\clearpage
\section{}
Zmontowano sumator o dwóch wejściach.

% Zsumować drgania sinusoidalne z dwóch generatorów, obserwować zdudnienia przebiegów.

\begin{align}
    R_1 &= 19.77k\Omega \\
    R_2 &= 1.99k\Omega \\
    R_3 &= 2.01k\Omega \\
\end{align}
\begin{align}
    U_{wy} = -\;\frac{R_2}{R_1}\;U_{we}
\end{align}

\begin{figure}[H]
	\centering
	\begin{circuitikz}[european] 
        \draw (0, 0) node[op amp] (opamp) {};
        
        % \draw (opamp.-) to[short, -*] ++(-2,0)
        % coordinate (nNode)
        % node[anchor=east] {$U_{we}$};
        
        \draw (opamp.-) to[short,*-] ++(0,1)
        coordinate (leftR)
        to[R, l=$R_2$] (leftR -| opamp.out)
        to[short,-*] (opamp.out)
        to[short,-*] ++(1,0)
        node[anchor=west] {$U_{wy}$};
        
        \draw (opamp.+)
        to[short,-] ++(0,-1)
        node[ground](GND){};
	\end{circuitikz}
	\caption{Schemat sumatora o dwóch wejściach}
\end{figure}
