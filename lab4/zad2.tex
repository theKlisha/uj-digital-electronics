\section{}
Zamontowano układ scalony 7400 w gnieździe płytki UC-1 a następnie zbadano tablicę logiczną dla zawartych w nim bramek logicznych NAND.
Powyższe kroki powtórzono dla układu 7402 (NOR).
Ze względu na brak dostępności układ 7486 (XOR) nie został zbadany.

\begin{table}[H]
    \centering
    \begin{tabular}{c c | c c || c c}
        \hline
        \(U_A\)[V] & stan logiczny A & \(U_B\)[V] & stan logiczny B & \(U_Y\)[V] & stan logiczny Y
        \\ \hline\hline
        0          & 0               & 0          & 0               & 3.96       & 1               \\ \hline
        0          & 0               & 5.01       & 1               & 3.96       & 1               \\ \hline
        5.01       & 1               & 0          & 0               & 3.96       & 1               \\ \hline
        5.01       & 1               & 5.01       & 1               & 83m        & 0               \\ \hline
    \end{tabular}
    \caption{Wyniki badania bramki NAND w układzie 7400 (UCY7400), pomiary odpowiadają informacją zawartym w nocie katalogowej.}
\end{table}

\begin{table}[H]
    \centering
    \begin{tabular}{c c | c c || c c}
        \hline
        \(U_A\)[V] & stan logiczny A & \(U_B\)[V] & stan logiczny B & \(U_Y\)[V] & stan logiczny Y
        \\ \hline\hline
        0          & 0               & 0          & 0               & 3.96       & 1               \\ \hline
        0          & 0               & 5.01       & 1               & 162m       & 0               \\ \hline
        5.01       & 1               & 0          & 0               & 163m       & 0               \\ \hline
        5.01       & 1               & 5.01       & 1               & 161m       & 0               \\ \hline
    \end{tabular}
    \caption{Wyniki badania bramki NOR w układzie 7402 (74LS02), pomiary odpowiadają informacją zawartym w nocie katalogowej.}
\end{table}
