\section{}
Z bramek NAND w układzie scalonym 7400 zmontowano przerzutnik asynchroniczny RS.
Sprawdzono tabelę przejść przerzutnika i porównano z teoretyczną.
Zmontowany układ działał zgodnie z oczekiwaniami.

\begin{figure}[H]
    \centering
    \begin{circuitikz}
        \draw
        (0, 1.5) node(gate1)[nand port] {}
        (0, -1.5) node(gate2)[nand port] {};

        \draw
        (gate1.in 1) to[short,-*] ++(-1, 0)
        node[anchor=east] {R}
        (gate2.in 2) to[short,-*] ++(-1, 0)
        node[anchor=east] {S}
        (gate1.out) to[short,-*] ++(2, 0)
        node[anchor=west] {Q}
        (gate2.out) to[short,-*] ++(2, 0)
        node[anchor=west] {$\overline{Q}$};

        \draw
        (gate1.in 2) to[short, -] ++(-0.5, 0)
        to[short, -] ++(0, -0.7)
        to[short, -] (1, -0.5)
        to[short, -*] (1, |- gate2.out);

        \draw
        (gate2.in 1) to[short, -] ++(-0.5, 0)
        to[short, -] ++(0, 0.7)
        to[short, -] (1, 0.5)
        to[short, -*] (1, |- gate1.out);
    \end{circuitikz}
    \caption{Schemat przerzutnika asynchronicznego RS zbudowanego z bramek NAND.}
\end{figure}