\section{}
Używając bramek logicznych NAND (7400), NOR (7402) zbudowano układ realizujący iloczyn logiczny, sumę logiczną oraz funkcję negacji.
Sprawdzono poprawność tablicy logicznej zbudowanych układów.

\begin{figure}[H]
    \centering
    \begin{circuitikz}
        \draw
        (0, 0) node(gate1)[nand port] {}
        (2, 0) node(gate2)[nand port] {};

        \draw
        (gate1.in 1) to[short,-*] ++(-1, 0)
        node[anchor=east] {A};

        \draw
        (gate1.in 2) to[short,-*] ++(-1, 0)
        node[anchor=east] {B};

        \draw
        (gate2.out) to[short,-*] ++(1, 0)
        node[anchor=west] {Y};

        \draw
        (gate1.out) to[short,-*] (gate2.in 1 |-, 0)
        coordinate (center)
        (center) to[short,-] (gate2.in 1)
        (center) to[short,-] (gate2.in 2);
    \end{circuitikz}
    \caption{Schemat układu zbudowanego z bramek NAND realizującego iloczyn logiczny.}
\end{figure}

\begin{figure}[H]
    \centering
    \begin{circuitikz}
        \draw
        (0, 1) node(gate1)[nand port] {}
        (0, -1) node(gate2)[nand port] {}
        (2, 0) node(gate3)[nand port] {};

        \draw
        (gate1.in 1) to[short,-] (gate1.in 1 |-, 1)
        (gate1.in 2) to[short,-] (gate1.in 2 |-, 1)
        to[short,*-*] ++(-1, 0)
        node[anchor=east] {A};

        \draw
        (gate2.in 1) to[short,-] (gate2.in 1 |-, -1)
        (gate2.in 2) to[short,-] (gate2.in 2 |-, -1)
        to[short,*-*] ++(-1, 0)
        node[anchor=east] {B};

        \draw
        (gate1.out) to[short,-] (gate3.in 1 |- gate1.out)
        to[short,-] (gate3.in 1);

        \draw
        (gate2.out) to[short,-] (gate3.in 2 |- gate2.out)
        to[short,-] (gate3.in 2);

        \draw
        (gate3.out) to[short,-*] ++(1, 0)
        node[anchor=west] {Y};
    \end{circuitikz}
    \caption{Schemat układu zbudowanego z bramek NAND realizującego sumę logiczną.}
\end{figure}

\begin{figure}[H]
    \centering
    \begin{circuitikz}
        \draw
        (0, 0) node(gate1)[nor port] {}
        (2, 0) node(gate2)[nor port] {};

        \draw
        (gate1.in 1) to[short,-*] ++(-1, 0)
        node[anchor=east] {A};

        \draw
        (gate1.in 2) to[short,-*] ++(-1, 0)
        node[anchor=east] {B};

        \draw
        (gate2.out) to[short,-*] ++(1, 0)
        node[anchor=west] {Y};

        \draw
        (gate1.out) to[short,-*] (gate2.in 1 |-, 0)
        coordinate (center)
        (center) to[short,-] (gate2.in 1)
        (center) to[short,-] (gate2.in 2);
    \end{circuitikz}
    \caption{Schemat układu zbudowanego z bramek NOR realizującego sumę logiczną.}
\end{figure}

\begin{figure}[H]
    \centering
    \begin{circuitikz}
        \draw
        (0, 1) node(gate1)[nor port] {}
        (0, -1) node(gate2)[nor port] {}
        (2, 0) node(gate3)[nor port] {};

        \draw
        (gate1.in 1) to[short,-] (gate1.in 1 |-, 1)
        (gate1.in 2) to[short,-] (gate1.in 2 |-, 1)
        to[short,*-*] ++(-1, 0)
        node[anchor=east] {A};

        \draw
        (gate2.in 1) to[short,-] (gate2.in 1 |-, -1)
        (gate2.in 2) to[short,-] (gate2.in 2 |-, -1)
        to[short,*-*] ++(-1, 0)
        node[anchor=east] {B};

        \draw
        (gate1.out) to[short,-] (gate3.in 1 |- gate1.out)
        to[short,-] (gate3.in 1);

        \draw
        (gate2.out) to[short,-] (gate3.in 2 |- gate2.out)
        to[short,-] (gate3.in 2);

        \draw
        (gate3.out) to[short,-*] ++(1, 0)
        node[anchor=west] {Y};
    \end{circuitikz}
    \caption{Schemat układu zbudowanego z bramek NOR realizującego iloczyn logiczny.}
\end{figure}
