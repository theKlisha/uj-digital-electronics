\section{}
Wykorzystując płytkę UC-1 zbudowano układ z dwóch rejestrów przesuwnych: 74164 i 74165.
Szeregowe wejście rejestru 74164 podłączono do impulsatora, wyjścia równoległe rejestru podłączono do wskaźników LED w celu śledzenia stanu rejestru.
Wyjścia równoległe układu 74164 podłączono również do wejść równoległych układu 74165, a jego wyjście szeregowe do próbnika stanów logicznych na płytce UC-1.
Utworzony w ten sposób szereg (impulsator, 74164, 74165, wskaźnik logiczny) testowano następująco.

Na wejście CLK rejestru 74164 podawano sygnał zegarowy, podając jednocześnie impulsy (z użyciem impulsatora) na wejście szeregowe.
Za pośrednictwem diod LED obserwowano stan rejestru przesuwnego: bity przesuwały się o jeden ilekroć układ wykrywał narastające zbocze sygnału zegarowego.
Najmłodszy bit odzwierciedlał stan impulsatora na wejściu szeregowym w momencie przesunięcia rejestru.

Następnie chcąc przetestować działanie układu 74165, ustalono stan na magistrali równoległej pomiędzy dwoma rejestrami impulsując na wejście 74164, przesuwając go odpowiednio.
W kolejnym kroku podano stan niski na wejście \(SH/\overline{LD}\) od 74165, co spowodowało zapis stanu ustalonego na magistrali równoległej do rejestru.
Zmieniono stan wejścia \(SH/\overline{LD}\) na wysoki, umożliwia to przesunięcie rejestru 74165 na skutek narastającego zbocza sygnału zegarowego, oraz jednoczesne uniewrażliwienie go na zmianę stanu magistrali.
Podawano następnie sygnał zegarowy obserwując jednocześnie stan wskaźnika logicznego na wyjściu szeregowym rejestru 74165.
Z kolejnymi narastającymi zboczami na wejściu CLK rejestr przesuwał się ustawiając stan wyjścia szeregowego na kolejne bity zapisane uprzednio na magistrali równoległej.