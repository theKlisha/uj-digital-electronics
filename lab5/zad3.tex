\section{}
Zbadano działanie zawartego w układzie 7475, wyzwalanego poziomem przerzutnika D
Sygnał zegarowy oraz sygnał na wejście informacyjne podano za pomocą impulsatora znajdującego się na płytce UC-1.

Zaobserwowane działanie przerzutnika było zgodne z oczekiwanym. W przeciwieństwie do przerzutnika D wyzwalanego zboczem, stan wyjścia badanego układu odzwierciedlał stan wejścia informacyjne tak długo jak stan wejścia zegarowego był wysoki.

\begin{table}[H]
    \centering
    \begin{tabular}{c|c||c|c}
        \hline
        zegar & d     & \(Q_{n+1}\) & \(\overline{Q_{n+1}}\)
        \\ \hline\hline
        1     & 0     & 0           & 1                      \\ \hline
        1     & 1     & 1           & 0                      \\ \hline
        0     & \(-\) & \(Q_n\)     & \(\overline{Q_n}\)     \\ \hline
    \end{tabular}
    \caption{Tabela prawdy badanego przerzutnika D.}
\end{table}