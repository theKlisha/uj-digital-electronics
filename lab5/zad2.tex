\section{}
Zbadano działanie zawartego w układzie 7474, wyzwalanego zboczem narastającym przerzutnika D
Sygnał zegarowy podano za pomocą impulsatora znajdującego się na płytce UC-1.
Zaobserwowane działanie przerzutnika było zgodne z oczekiwanym.

\begin{table}[H]
    \centering
    \begin{tabular}{c|c||c|c}
        \hline
        zegar               & d     & \(Q_{n+1}\) & \(\overline{Q_{n+1}}\)
        \\ \hline\hline
        0 \(\rightarrow\) 1 & 0     & 0           & 1                      \\ \hline
        0 \(\rightarrow\) 1 & 1     & 1           & 0                      \\ \hline
        0                   & \(-\) & \(Q_n\)     & \(\overline{Q_n}\)     \\ \hline
        1                   & \(-\) & \(Q_n\)     & \(\overline{Q_n}\)     \\ \hline
    \end{tabular}
    \caption{Tabela prawdy badanego przerzutnika D.}
    % \\[\baselineskip] (*) Ustalony niski lub wysoki stan logiczny}
\end{table}