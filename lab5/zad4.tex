\section{}
Wykorzystując zawarty w układzie 7593 przerzutnik JK zbudowano układ dzielący częstotliwość przez dwa.
Dokonano tego łącząc ze sobą wejścia J i K przerzutnika JK otrzymując w ten sposób przerzutnik T.

Przerzutniki w układzie 7593 wyzwalane są zboczem opadającym, które to występuje raz na okres sygnału prostokątnego.
Negacja wyjścia występować więc będzie raz na okres sygnału wejściowego, co daje jeden okres sygnału wyjściowego na każde dwa okresy sygnału wejściowego. Jest to równoznaczne z podziałem częstotliwości na dwa.

\begin{table}[H]
    \centering
    \begin{tabular}{c|c||c|c}
        \hline
        zegar               & d & \(Q_{n+1}\)        & \(\overline{Q_{n+1}}\)
        \\ \hline\hline
        \(-\)               & 0 & \(Q_n\)            & \(\overline{Q_n}\)     \\ \hline
        1 \(\rightarrow\) 0 & 1 & \(\overline{Q_n}\) & \(Q_n\)                \\ \hline
    \end{tabular}
    \caption{Tabela prawdy zbudowanego przerzutnika T.}
\end{table}